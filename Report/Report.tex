\documentclass{article}
\usepackage[utf8]{inputenc}
\usepackage[margin=0.9in]{geometry}
\usepackage[colorlinks]{hyperref}
\begin{document}
\title{\textbf{Natural Language Processing - CS6370 }
\\
\textbf{Spell Check Assignment Report}}
\author{Aravind Sankar CS11B033 \\
Sriram V CS11B058 \\
\\[0.2in]
}

\maketitle
\section{Introduction}
This assignment involved designing an efficient spell checker. This spell checking assignment was divided into three parts : 
\begin{itemize}
\item Word checker where spelling corrections are given for a misspelled word not present in the dictionary.
\item Phrase checker where spelling corrections are given for misspelled word(s) present in phrases.
\item Sentence checker where spelling corrections are given misspelled word(s) present in sentences.

The Spell Checker was implemented in \textit{Python}.
\end{itemize}

\section{Corpora used :}
\begin{itemize}

\item The Unix dictionary for american english, which has close to 73,000 valid english words was used as the dictionary to identify if a given word has a spelling error.

\item Corpus of Contemporary American English (COCA) dataset, which has 1,000,000 most frequent n-grams. This was used in the phrase and sentence spell checkers.	

\item The brown corpus was also initially used for learning context words, but isn't part of the final spell checker model.

\end{itemize}

\section{Word Checker}
The basic idea of approach for Word Checker was obtained from the paper by Kernighan et.al which was titled - A Spelling Correction Program Based on a Noisy Channel Model. This paper doesn't take the context in which a misspelt word appears into account, and provides spelling corrections for standalone words.



\end{document}
